\section{Conclusão}

Como tem-se observado ao longo dos últimos 6 anos, \emph{Drones
} são uma tecnologia que encontra cada vez mais espaço no 
cotidiano das pessoas, principalmente no exterior. Com o 
processo de regulamentação das aeronaves ao redor do mundo, a
tecnologia tem se difundido lentamente, adquirindo uma força
crescente.

Atualmente os países que lideram a produção dessa tecnologia
são China, no ramo civil e Estados Unidos, no ramo militar
(enfatiza-se que este segmento já está em desenvolvimento desde
a metade do século XX). Porém, as aplicações civis estão apenas
começando a ser descobertas, sem grandes empresas atuando 
nacional e internacionalmente no ramo. 

É comum encontrar pequenas empresas brasileiras que sobrevivem
aplicando \emph{Drones} de empresas como a DJI no ramo da 
fotografia, dado o custo relativamente baixo da plataforma e 
sua facilidade de uso. Porém, como dito anteriormente, tais 
tecnologias são fechadas e focadas em uso amador, limitando 
suas aplicações.

Tendo em vista os fatos levantados, esse documento se propôs
a descrever a estrutura completa de uma empresa brasileira que
desenvolveria seu próprio controlador de voo para drones,
permitindo a comercialização deste no mercado \emph{hobbysta} 
nacional com preços mais atrativos quando comparados aos de 
concorrentes externos e possibilidade de assistência em 
território nacional, além de sua utilização na montagem de \
\emph{Drones} para prestação de serviço customizada ao desejo 
do cliente. Isso é extremamente facilitado pelo fato de que a 
empresa detém a tecnologia de controle de voo, diferentemente 
das concorrentes, que utilizam tais plataformas fechadas 
e importadas.

Após as análises realizadas, conclui-se que a empresa é viável
e exploraria um mercado praticamente desconhecido no país, 
cujos potenciais são enormes. O Brasil se apoia expressivamente
na produção agrícola, a qual envolve enormes quantias de terra 
e de circulação de capital. Como apresentado anteriormente, há
inúmeras aplicações potenciais de \emph{Drones} nessa área, as 
quais, com pouco esforço ou necessidade de desenvolvimento 
tecnológico (tendo o controlador de voo funcional), gerariam
ganhos significativos para os produtores, tornando a 
contratação dos serviços oferecidos pela empresa um enorme
atrativo.


