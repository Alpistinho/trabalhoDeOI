\section{Plano de Gestão}

O objetivo dessa seção é estruturar com maior clareza a estratégia de atuação da empresa,
levando em conta micro e macroaspectos que influenciam em seu desempenho. Para tanto, serão 
feitas duas análises: CAMGPEST e fatores críticos de sucesso. A primeira análise consiste num 
diagnóstico dos macrofatores culturais, ambientais, mercadológicos, geográficos, políticos, 
econômicos, sociais e tecnológicos que dizem respeito ao escopo da empresa. Já o segundo método 
realiza um diagnóstico dos pontos fortes e fracos dos fatores mais relevantes aos principais 
concorrentes.

\subsection{Análise CAMGPEST}

\subsubsection*{Fatores Culturais}

Drones têm entrado cada vez mais na cultura popular ao redor do mundo. Até mesmo no Brasil, 
onde a tecnologia costuma chegar com certo atraso, observam-se estatísticas de compras crescentes, 
principalmente no ramo de drones "populares", ou seja, prontos para voar \emph{out-of-the-box}, porém com 
aplicações limitadas à obtenção de imagens aéreas amadoras e entretenimento \cite{brasildrones}. 
O exemplo mais notório desse tipo de drones é, 
certamente, o Phantom, atualmente em sua quarta iteração, da concorrente chinesa DJI.

A difusão global desse tipo de aeronaves, respaldada por taxas de aumentos de vendas de drones mundiais 
como a entre os anos de 2014 e 2015, quando o aumento foi de 63\% \cite{dronestats}, certamente 
mostra-se benéfica para a empresa. Com um público que conhece drones cada vez maior, cria-se, 
primariamente, um ambiente propício à prestação de serviço com essa tecnologia e, secundariamente, 
um público hobbysta crescente.

\subsubsection*{Fatores Ambientais}

Como será descrito no Plano de Marketing, a tecnologia de Drones permite diversos impactos positivos 
no que diz respeito ao meio-ambiente. Como uma das áreas de prestação de serviços de maior potencial 
é a obtenção de imagens agrícolas aéreas, pode-se, consequentemente, monitorar processos de recuperação 
ambiental, avaliar danos ambientais e desmatamento, encontrar e delimitar incêndios de modo eficiente e 
monitorar pragas. Ademais, a fonte de energia das aeronaves é bastante mais vantajosa do que combustíveis 
fósseis, no que tange a emissão de poluentes, que é essencialmente nula, nesse caso.

\subsubsection*{Fatores Mercadológicos}

Clientes potenciais, sob o ponto de vista da empresa, são, primariamente, atuantes dos seguintes setores:

\begin{itemize}
	\item Agrícola: proprietários de fazendas e de grandes áreas que exigem qualquer tipo de inspeção
	\item Ambiental: institutos de licenciamento e gerenciamento de recursos ambientais e naturais, como o IBAMA
	\item Engenharia Civil: construtoras ou órgãos de defesa civíl, os quais se beneficiariam da obtenção de 
	imagens para inspeção
	\item Indústria Cinematográfica: todo o público interessado em formas de se obter imagens aéreas com custos reduzidos e 
	maior flexibilidade para fins de entretenimento
	\item Hobbystas: foco do controlador de voo, cuja vantagem há de ser a oferta de capacidades semelhantes às dos 
	concorrentes sob menor custo
\end{itemize}

\subsubsection*{Fatores Geográficos}

Uma das maiores vantagens da empresa advém da dificuldade (que se apresenta em forma de altos preços) 
de atuação de empresas do exterior, as quais usualmente detém esse tipo de tecnologia, no Brasil. Essa 
vantagem se expressa, principalmente, no setor de prestação de serviços com Drones, o qual é extremamente 
carente de atuantes internacionais. Quanto ao controlador de voo, pode-se praticar preços menores, isentos 
dos enormes impostos alfandegários praticados no país.

\subsubsection*{Fatores Políticos}

Devido à natureza da empresa, sua atuação não expressará impactos políticos importantes além de, 
possivelmente, fomentar desenvolvimento tecnológico em território nacional, também com a venda dos controladores de voo.

\subsubsection*{Fatores Econômicos}

A atuação da empresa se vê beneficiada pela situação econômica do país de duas formas, levando em 
conta a atual situação de recessão. 

Primeiramente, quanto à prestação de serviços, os clientes potencialmente aumentarão seus lucros 
ao utilizarem-se da tecnologia oferecida. Um exemplo disso é na agricultura: com a obtenção de 
imagens aéreas por meio de drones (cujo custo de operacional é bastante menor que o de helicópteros 
e satélites), um produtor pode saber, de forma bastante precisa, o número de unidades plantadas e, 
consequentemente, a quantia de pesticida a ser utilizada sem grandes desperdícios. Além disso, uma 
construtora pode inspecionar certa área de difícil acesso de uma obra sem ter de colocar um operário 
em atividade de risco, o que acarreta altos custos.

Secundariamente, no ramo das vendas do controlador de voos, temos, outra vez, o baixo custo como 
principal vantagem. Além disso, com o crescimento da classe média observado ao longo dos últimos 
dez anos, pode-se esperar um público hobbysta expressivo com interesses no produto.

\subsubsection*{Fatores Sociais}

Devido à natureza da empresa, não esperam-se impactos sociais expressivos.

\subsubsection*{Fatores Tecnológicos}
	
Os objetivos da empresa são altamente correlacionados à tecnologia. Podem-se enumerar diversas influências 
positivas da área na produção do controlador de voo e na prestação de serviços. 

Quanto ao controlador de voo, é notória a importância do desenvolvimento tecnológico. A crescente 
miniaturização de componentes com desempenho de processamento cada vez melhor, aliada à disponibilidade 
de sensores cada vez mais precisos e confiáveis permite o desenvolvimento de controladores cujo desempenho 
viabiliza estabilidade excepcional, mantendo a massa da aeronave mínima, permitindo que o usuário adicione 
outros dispositivos, conforme a aplicação almejada. Uma área que tem crescido muito ao longo da última década, 
a Inteligência Artificial, potencialmente contribuirá muito no controle de voo de aeronaves, permitindo 
tomada de decisões autônoma e uso mais eficiente de recursos energéticos.

A prestação de serviços será mais eficiente à medida que o controlador de voo for mais eficiente. Porém, 
essa área também será beneficiada por tecnologias periféricas. Um exemplo disso é o desenvolvimento de 
aplicativos para tablets, permitindo interação e visualizações rápidas dos dados coletados pelo drone com 
mapas de relevo 3D, georreferenciamento com coordenadas de GPS etc.

\subsection{Análise de Fatores Críticos de Sucesso}

Essa análise utiliza um sistema de pontuação para comparar a empresa aos seus principais concorrentes de 
forma a identificar suas vantagens competitivas e desvantagens que precisam ser trabalhadas.

Como a empresa se propõe a atender dois mercados bastantes distintos, duas comparações serão realizadas, 
uma em relação ao mercado hobbysta e outra em relação à prestação de serviços.

A seguir, a primeira tabela é apresentada com os principais fatores críticos de sucesso em relação ao 
mercado de serviços:

\begin{table}[!htbp]
	\centering
	\begin{tabular}{ll|l|l|l|l|}
		\cline{3-6}
		&      & \multicolumn{2}{l|}{Placeholder} & \multicolumn{2}{l|}{Prestadores de serviço} \\ \hline
		\multicolumn{1}{|l|}{Pontos fortes}             & Peso & Nota         & Desempenho        & Nota              & Desempenho              \\ \hline
		\multicolumn{1}{|l|}{Preço}                     & 15   & 8            & 120               & 6                 & 90                      \\ \hline
		\multicolumn{1}{|l|}{Flexibilidade de serviços} & 20   & 10           & 200               & 6                 & 120                     \\ \hline
		\multicolumn{1}{|l|}{Proximidade ao cliente}    & 20   & 10           & 200               & 10                & 200                     \\ \hline
		\multicolumn{1}{|l|}{Acesso ao hardware}        & 20   & 10           & 200               & 0                 & 0                       \\ \hline
		\multicolumn{1}{|l|}{Total}                     &      &              & 720               &                   & 410                     \\ \hline
	\end{tabular}
	\label{pontoForteServico}
	\caption{Pontos Fortes no Mercado de Prestação de Serviço}
\end{table}


\begin{table}[!htbp]
	\centering
	\resizebox{\textwidth}{!}{
	\begin{tabular}{ll|l|l|l|l|}
		\cline{3-6}
		&      & \multicolumn{2}{l|}{Placeholder} & \multicolumn{2}{l|}{Prestador de serviço} \\ \hline
		\multicolumn{1}{|l|}{Pontos fracos}                                 & Peso & Nota         & Desempenho        & Nota             & Desempenho             \\ \hline
		\multicolumn{1}{|l|}{Reconhecimento no mercado}                     & 15   & 0            & 0                 & 10               & 150                    \\ \hline
		\multicolumn{1}{|l|}{Variedade de produtos oferecidos}              & 15   & 5            & 75                & 10               & 150                    \\ \hline
		\multicolumn{1}{|l|}{Atendimento de múltiplos clientes simultâneos} & 10   & 5            & 50                & 8                & 80                     \\ \hline
		\multicolumn{1}{|l|}{Equipamento de alto desempenho}                & 10   & 5            & 50                & 8                & 80                     \\ \hline
		\multicolumn{1}{|l|}{Total}                                         &      &              & 175               &                  & 460                    \\ \hline
	\end{tabular}
	}
	\caption{Pontos Fracos no Mercado de Prestação de Serviços}
	\label{pontoFracoServico}
\end{table}

Pode-se observar que a respeito do mercado de prestação de serviço a capacidade da empresa desenvolver 
o hardware dos drones utilizados \emph{in-house} a permite oferecer preços mais baixos do que os concorrentes 
que utilizam drones comerciais. Além disso, o acesso ao hardware permite que a empresa ofereça serviços mais 
personalizados a diferentes clientes, tendo a capacidade de alterar o comportamento do drone sob medida. 
Por outro lado, por ser uma nova ingressante no mercado, a marca da empresa não possuirá reconhecimento. 
Ademais, uma empresa pequena não será capaz de oferecer serviços a muitos clientes simultâneos nem equipamentos 
de tão alto desempenho quanto os desenvolvidos pelas empresas líderes do mercado.

A tabela abaixo descreve os fatores críticos de sucesso em relação ao mercado hobbysta:

\begin{table}[!htbp]
	\centering

	\begin{tabular}{ll|l|l|l|l|}
		\cline{3-6}
		&      & \multicolumn{2}{l|}{Placeholder} & \multicolumn{2}{l|}{Fabricantes tradicionais} \\ \hline
		\multicolumn{1}{|l|}{Pontos fortes}             & Peso & Nota         & Desempenho        & Nota               & Desempenho               \\ \hline
		\multicolumn{1}{|l|}{Preço}                     & 15   & 7            & 105               & 6                  & 90                      \\ \hline
		\multicolumn{1}{|l|}{Flexibilidade de serviços} & 15   & 8            & 120               & 6                  & 90                      \\ \hline
		\multicolumn{1}{|l|}{Proximidade ao cliente}    & 20   & 10           & 200               & 10                 & 200                      \\ \hline
		\multicolumn{1}{|l|}{Acesso ao hardware}        & 20   & 10           & 200               & 0                  & 0                       \\ \hline
		\multicolumn{1}{|l|}{Total}                     &      &              & 625               &                    & 380                      \\ \hline
	\end{tabular}
	\label{pontoForteHobby}
	\caption{Pontos Fortes no Mercado de \emph{Hobbysta}}
\end{table}

\begin{table}[!htbp]
	\centering
	\resizebox{\textwidth}{!}{
	\begin{tabular}{ll|l|l|l|l|}
		\cline{3-6}
		&      & \multicolumn{2}{l|}{Placeholder} & \multicolumn{2}{l|}{Fabricantes tradicionais} \\ \hline
		\multicolumn{1}{|l|}{}                                              & Peso & Nota         & Desempenho        & Nota               & Desempenho               \\ \hline
		\multicolumn{1}{|l|}{Reconhecimento}                                & 15   & 0            & 0                 & 10                 & 150                      \\ \hline
		\multicolumn{1}{|l|}{Variedade de produtos oferecidos}              & 15   & 4            & 60                & 8                  & 120                      \\ \hline
		\multicolumn{1}{|l|}{Atendimento de múltiplos clientes simultâneos} & 10   & 4            & 40                & 10                 & 100                      \\ \hline
		\multicolumn{1}{|l|}{Equipamento de alto desempenho}                & 10   & 5            & 50                & 10                 & 100                      \\ \hline
		\multicolumn{1}{|l|}{Total}                                         &      &              & 150               &                    & 470                      \\ \hline
	\end{tabular}
	}
	\caption{Pontos Fracos do Mercado de \emph{Hobbysta}}
	\label{my-label}
\end{table}

A respeito do mercado hobbysta, a produção nacional permite à empresa produzir um equipamento com preço 
mais baixo, com qualidade similar ou pouco inferior ao importado. No caso de um hobbysta mais dedicado, 
há ainda a possibilidade, proporcionada pela proximidade no mercado nacional, do desenvolvimento de soluções 
personalizadas.

Novamente, o reconhecimento no mercado é uma fragilidade inicial que precisa ser considerada. Além disso, 
os fabricantes tradicionais oferecem uma gama maior de produtos, tem maior capacidade produtiva e são 
capazes de desenvolver equipamentos de mais alto desempenho.