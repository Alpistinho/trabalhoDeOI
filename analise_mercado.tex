\section{Análise de Mercado}

\subsection{Estudo dos Clientes}

A empresa atenderia, potencialmente, pessoas físicas e jurídicas, dividindo a possível clientela em termos
dos dois principais produtos: o controlador de voo e a prestação de serviços.

\subsubsection*{Controlador de Voo}

Provavelmente apenas pessoas físicas de um nicho bastante específico: \emph{hobbystas}. Vender um controladorde voo com sua máxima capacidade de atuação e customização seria ingênuo, já que permitiria que terceiros utilizassem-no para prestar serviços, o que reduziria nossa fatia de clientes. Considerando-se isso, desejamos vender uma versão do controlador que permita fazer o mínimo necessário para voos recreativos e filmagens simples, de forma a competir com os produtos de empresas estrangeiras no que diz respeito a preços. Esse mercado já possui elevado potencial, segundo nossas pesquisas.

O controlador de voo seria direcionado a \emph{hobbystas} pois exige conhecimento razoavelmente avançado para ser utilizado. Não é um produto que funcionará \emph{out of the box}, dado que é necessário montar todo o resto do multirotor, acoplar o controlador a ele e calibrá-lo. Esse público é composto, majoritariamente, por homens de faixa etária acima de 15 anos, com média de aproximadamente 30 anos. Vale ressaltar que o limite inferior de faixa etária tende a diminuir, dado o crescente e fácil acesso a informações bastante técnicas e específicas disponibilizado pela internet.

Quanto ao ramo de atuação da clientela, a probabilidade é que encontremos mais pessoas inclinadas ao lado das disciplinas exatas, devido ao fator técnico do produto. Porém, o público não é restrito a profissinais dessa área. Encontramos, em nossas pesquisas, pessoas da área do Direito e da Medicina, por exemplo. Fato é que o poder aquisitivo do público alvo tem de ser razoavelmente alto, dados os preços de montagem e manutenção dessas plataformas. Um ponto interessante é que \emph{hobbystas} tendem a levar a prática bastante a sério, tornando-se clientes assíduos e leais, dispostos a pagar quantias altas ao encontrarem produtos que julguem interessantes. Falamos isso também por experiência própria. O nível de escolaridade da clientela tende, também, a ser alto: provavelmente, no mínimo, Ensino Médio Técnico completo. Clientes desse tipo são encontrados em todo o país, principalmente nas grandes cidades, como Rio de Janeiro ou São Paulo.

A probabilidade de um \emph{hobbysta} comprar o produto mais do que uma vez não é tão alta, porém não é nula. Por outro lado, reiterações do produto (versões novas) tendem a atrair o público, que é leal, como citado anteriormente. Atualmente, esse público se dispõe a pagar de R\$ 800.00 a 1200.00 nesse componente (modelo Naza V2 da concorrente DJI). As vendas podem ser feitas \emph{online}, com envio pelos \emph{Correios}, já que o produto é bastante pequeno.

\subsubsection*{Prestação de Serviços}

A prestação de serviços é o produto (pode chamar assim?) que atingirá o maior público alvo, também gerando a maior parcela de faturamento da empresa. Como dito anteriormente, aplicações para \emph{Drones} surgem a cada dia, potencialmente otimizando e reduzindo custos de certas atividades.

Aqui, o maior faturamento virá provavelmente de prestação de serviços à empresas. A ideia é adaptar o controlador de voo base para uma versão customizada a cada cliente. O controlador básico permite que a aeronave voe de forma estável e autônoma, com rotas pré-programadas ou controle manual. Com isso, podemos, por exemplo, permitir transmissão de imagem em tempo real e \emph{logging} de coordenadas de GPS para mapear a área de um terreno agrícola. Outra aplicação seria coletar dados de forma remota: pode-se acoplar quaisquer sensores à aeronave, permitindo, por exemplo, que uma empresa que deseja instalar painéis solares navegue com o \emph{Drone} ao longo do terreno, medindo a incidência média de luz do sol (e outras variáveis como temperatura, umidade etc) e, em seguida (ou em tempo real), obtenha um mapa com os dados explicitados, possibilitando escolher pontos que maximizem a incidência para realizar a instalação.

Empresas que trabalhem com engenharia civil, inspeções em geral, fotografia, filmagem dentre outras áreas também se beneficiariam da tecnologia. Pessoas físicas provavelmente se utilizariam de aplicações semelhantes às empresas porém em menor escala, de forma que as quantias de dinheiro envolvidas serão menores. Os serviços terão frequência de "reincidência"\ variável: é mais provável que pessoas físicas demandem mais filmagens de casamentos do que empresas demandem coletas de incidência solar, por exemplo. Por outro lado, o projeto de uma empresa tem nível de complexidade maior, exigindo \emph{software} adicional para interação com a aeronave, assistência e talvez a venda da aeronave, especializada para certa aplicação, como um todo, junto à assistência de como utilizá-la. Preços de serviços de fotografia ou filmagem podem variar de R\$ 200.00 a 2000.00, para grandes eventos, com possíveis transmissões ao vivo, diversos ângulos, câmeras melhores etc. Serviços diversos, na área de mapeamento ou sensoriamento, com ou sem a venda da plataforma, podem variar de R\$ 1000.00 a 30000.00 ou mais, dependendo do projeto.

A depender da magnitude do projeto, pode haver deslocamento da equipe para prestação de serviço, porém almejamos iniciar com serviços na região do Rio de Janeiro, onde há potencial demanda, dado o tamanho da cidade.

\subsection{Estudo dos Concorrentes}

O principal concorrente é certamente a chinesa DJI, com seu famoso \emph{Drone} Phantom, atualmente na quarta iteração. De qualidade excepcional, oferece grande facilidade de uso, tempo de voo acima da média, estabilidade excelente e filmagens e fotos de ótima qualidade. A empresa também oferece seu controlador de voo, o Naza V2 para entusiastas e \emph{hobbystas} que desejam a mesma estabilidade numa plataforma diferente. Como o produto é de origem chinesa, é possível encontrá-lo em terras brasileiras, por meio de revendedores, por cerca de R\$5500.00 (Phantom) e R\$ 1000.00 (Naza V2). Mesmo com os altos preços, é notório o público que se dispõe a comprar o produto.

A nossa vantagem em comparação com a DJI é apostar na área de prestação de serviços. A plataforma da empresa chinesa é totalmente fechada, direcionada única e exclusivamente a um usuário final médio, com pouco conhecimento na área, disposto a adquirir um "brinquedo"\ avançado. Existem empresas que empregam o Phantom em prestação de serviços, porém de forma limitada à captura de imagens, que é o máximo oferecido pela DJI. Há uma crescente comunidade que visa modificar o Phantom para permitir novas capacidades, porém imaginamos que um produto criado com uma aplicação em mente dificilmente se adapta tão bem a outra aplicação.

Na área de prestação de serviços, há uma crescente onda de empresas brasileiras que utilizam \emph{Drones} comerciais (como o Phantom) para aplicações que envolvem captura de imagens. Porém, nenhuma dessas empresas detém tecnologia própria. Como mencionado anteriormente, o Brasil ainda é extremamente pobre no que diz respeito à geração de tecnologia nessa área. Isso limita as empresas brasileiras às aplicações pensadas originalmente pelas empresas do exterior, de forma que enxergamos aí um espaço para, junto ao cliente, encontrar uma solução maximamente otimizada na forma de um controlador de voo especializado. 

\subsection{Estudo dos Fornecedores}

Não só não existe produção de controladores de voo no Brasil, como tamém não existe produção dos outros componentes de um \emph{Drone}. Isso inclui a estrutura (essa é a mais facilmente substitutível por um produto brasileiro, já que não exige tecnologias muito avançadas), os motores elétricos, as baterias, os controladores de velocidade dos motores etc.

Isso implica que teremos que importar peças, por exemplo, através do \emph{AliExpress}. Como nosso intuito não é primariamente o de vender \emph{Drones} (e caso isso seja necessário, incluiremos o custo da plataforma no custo do serviço), teremos que manter algumas plataformas próprias, cujo número aumentará com o crescimento da empresa, porém não muito rapidamente. Um estoque de segurança deverá ser mantido, levando em conta o fato de que entregas pelo \emph{AliExpress} costumam levar meses.

Quanto ao controlador de voo, teremos que encontrar fornecedores de placas de circuito impresso para realizarem a impressão do nosso design e, talvez, a soldagem dos componentes. Para tanto, entraremos em contato com as empresas que desenvolvem os componentes do controlador (microcontrolador e sensores), como: Bosch, ATMEL, InvenSense, U-Blox etc e negociaremos compras em quantidades elevadas dos componentes, considerando tempo de entrega e custos.